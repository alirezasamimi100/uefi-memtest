
\def\myfigure#1#2{ \شروع{شکل}[ht]
\centerimg{#1}{14cm}
\شرح{#2}
\پایان{شکل}
}

\def\lrtt#1{\lr{\tt #1}}

\فصل{بوت امن}

بوت امن (\lr{Secure Boot}) یکی از ویژگی‌های امنیتی سیستم‌های \lr{UEFI} است که هدف آن جلوگیری از اجرای
کدهای غیرمجاز در فرآیند بوت سیستم می‌باشد. در این فرآیند، تمام نرم‌افزارهای بوت باید دارای امضای
دیجیتال معتبر باشند. ابزار \lr{shim} یک واسط است که به کاربران امکان می‌دهد گواهی‌های سفارشی یا کلیدهای
اضافی را برای بوت امن اضافه کنند، به خصوص در سیستم‌های لینوکسی که ممکن است امضاهای استاندارد
مایکروسافت را نداشته باشند. این روش باعث حفظ امنیت در عین انعطاف‌پذیری می‌شود.

از آنجا که خروجی \lr{EFI} ما امضای رسمی مایکروسافت را ندارد، به \lr{shim} برای اجرای آن در محیط بوت امن نیاز داریم.

\قسمت{شبیه‌سازی}

برای شبیه‌سازی بوت امن از \lr{Virtualbox} استفاده می‌کنیم. در این محیط می‌توانیم \lr{EFI} و بوت امن را
مشابه ماشین واقعی تست کنیم.  یک سیستم عامل مجازی اوبونتو نصب می‌کنیم تا عملیات‌های مرتبط با راه‌اندازی
برنامه را در آن انجام دهیم.

\قسمت{آماده‌سازی برنامه}

باید به برنامه \lr{section} به نام \lr{sbat} اضافه کنیم. مهم است مقدار آن را \lr{shim} بشناسد و به
همین خاطر از بخش مشابهی در \lr{grub} استفاده می‌کنیم.

\myfigure{sec/01}{اضافه کردن \lr{sbat}}

\قسمت{آماده‌سازی بوت}

برنامه \lr{shim} به طور پیشفرض در اوبونتو نصب می‌شود. لازم است فایل‌های \lr{shim} و برنامه \lr{EFI} را
به پارتیشن \lr{ESP} منتقل می‌کنیم. اسم برنامه را به \lrtt{grubx64.efi} تغییر می‌دهیم. به صورت
پیشفرض \lr{shim} برنامه‌ای با این اسم را لود می‌کند.

\myfigure{sec/02}{کپی کردن \lr{shim} به \lr{ESP}}
\myfigure{sec/03}{اضافه کردن \lr{boot option}}

\قسمت{بوت با اثرانگشت برنامه}

یک راه برای بوت امن این است که اجازه دهیم \lr{shim} بوت شود و سپس اثرانگشت برنامه را به \lr{MokList}
اضافه کنیم. بعد از انجام این کار \lr{shim} برنامه را اجرا می‌کند و خطای \lrtt{0x1a Security
Violation} چاپ نمی‌شود.  مراحل اضافه کردن اثرانگشت \lr{hash} در ادامه نشان داده شده است.

\myfigure{sec/06}{صفحه \lr{Mok Management}}
\myfigure{sec/07}{وارد کردن امضای برنامه}
\myfigure{sec/08}{منوی انتخاب برنامه‌ها}

\قسمت{بوت با امضای برنامه}

راه دیگر برای بوت امن این است که یک کلید شخصی بسازیم و برنامه \lr{EFI} را با آن امضا کنیم. بعد از
امضا کردن و اجرای دوباره \lr{shim}، می‌توانیم فایل \lr{DER} امضا را به \lr{MokList} اضافه کنیم. مزیت
این روش این است که اگر برنامه بعدا تغییر کند فقط لازم است یک بار دیگر آن را امضا کنیم و دیگر طی کردن
مراحل \lr{enroll key} لازم نیست.

\myfigure{sec/04}{ساخت کلید}
\myfigure{sec/05}{امضای برنامه \lr{EFI}}

\قسمت{منابع}

\begin{latin}
\url{https://wiki.archlinux.org/title/Unified_Extensible_Firmware_Interface/Secure_Boot}

\url{https://www.funtoo.org/UEFI_Secure_Boot_and_SHIM}

\url{https://github.com/rhboot/shim/blob/main/SBAT.md}

\url{https://github.com/rhboot/shim/issues/376}
\end{latin}
