
\فصل{نوشتن برنامه Hello World با EFI}

در این قسمت مراحل لازم برای نوشتن یک برنامه‌ی ساده‌ی Hello World را بیان می‌کنیم و در آخر هم کد زده شده را توضیح داده و نتیجه‌ی کار را هم ارائه می‌دهیم.

\قسمت{آماده کردن محیط برنامه نویسی EFI}

برای این قسمت ما از edk2 استفاده کرده‌ایم که می‌توانید آن را در این \href{https://github.com/tianocore/edk2}{لینک} مشاهده کنید.

\قسمت{تحلیل کد}

عکس کد این بخش را در ادامه می‌بینیم.
........................... عکس کد hello world ......................
حال به بررسی کد نوشته شده می‌پردازیم. همانطور که واضح است کد بسیار ساده می‌باشد و بسیار شبیه کد C برای این برنامه است. یک تابع main داریم و با دستور Print پیام مورد نظر را چاپ می‌کنیم.

\قسمت{اجرای کد}

....