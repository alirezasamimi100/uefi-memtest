
\فصل{آزمون حافظه}



\قسمت{آشنایی با آزمون حافظه}

آزمون حافظه به فرایند تست کردن و تایید کردن کارکرد، درستی و کارایی حافظه سیستم می‌گویند که می‌تواند شامل حافظه فیزیکی و یا حافظه مجازی شود. از این آزمون‌ها برای پیدا کردن خطاها، اعتبارسنجی حافظه فیزیکی، بررسی کارایی و تخصیص دادن حافظه استفاده 
کرد. آزمون حافظه‌ای که ما استفاده می‌کنیم از نوع Power-On Self-Test(POST) است یعنی 
هنگام Boot کامپیوتر، BIOS یک تست ساده را اجرا کند تا از کارایی حافظه مطمئن شود.

\قسمت{انواع آزمون‌های حافظه}

\زیرقسمت{Pattern Testing}

در این روش یک الگویی از داده‌ها را در حافظه می‌نویسیم و در آخر آن بخشها را می‌خوانیم و یکی بودنشان را بررسی می‌کنیم.

\زیرقسمت{Stress Testing}

برای شبیه‌سازی دنیای واقعی حافظه را  با خواندن و نوشت نبه شدت لود می‌کنیم تا پایداری آن را در این شرایط بسنجیم.

\قسمت{پیاده‌سازی آزمون حافظه}

برای پیاده‌سازی آزمون حافظه در ابتدا با متغیر pattern که توان‌هایی از 2 است را در خانه‌هایی از حافظه می‌نویسیم و سپس مقادیر همان خانه‌ها را می‌خوانیم و یکی بودنشان را بررسی می‌کنیم این کلیت آزمون اول یعنی آزمون WalkingOnesTest می‌باشد.

در ادامه کد تستی برای حالت چندپردازه خواهیم داشت که در آن بنابر آیدی پردازه بررسی می‌کنیم که در صفحه مربوط به آن پردازه هستیم یا نه و سپس در هر خانه م نظر آدرس آن خانه را می‌نویسیم و در آخر دوباره همه‌ی این مقادیر را می‌خوانیم تا از صحت اطلاعات مطمئن شویم.

